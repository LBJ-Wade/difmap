\documentclass[12pt]{article}

\begin{document}

\def\pair{\stackrel{\mbox{\tiny FT}}{\Longleftrightarrow}}
\def\ft{{\cal F}}
\def\ift{{\cal F}^{\mbox{\,-}1}}
\parindent 0pt

\section{Smoothing spectra via Fourier transforms}

Smoothing data with an even smoothing function is equivalent to
convolving the data with that function. This can be achieved via
Fourier transforms.
 
\subsection{Definition of terms}

\begin{itemize}
\item Let $\ft(g)$ denote the Fourier transform of a function $g$.
\item Let $\ift(G)$ denote the inverse Fourier transform of a function $G$.
\item Let $.$ denote multiplication, and $\otimes$ denote convolution.
\item Let the smoothing function be denoted by $s$.
\item Let the original arrays of data values and weights be denoted by $f$ and $w$, respectively.
\item Let the smoothed arrays of data values and weights be denoted by $\overline{f}$ and $\overline{w}$, respectively.
\end{itemize}

\subsection{Smoothing a discrete array using an unweighted convolution}

The Fourier convolution theorem says that multiplying the Fourier
transforms of two continuous functions by each other and then taking
the inverse transform of the result yields the convolution of the two
functions.

\begin{eqnarray}
     \overline{f} &=& f \otimes s \\
%
     \label{convolve}
     \overline{f} &=& \ift\left[\ft(f).\ft(s)\right]
\end{eqnarray}

For discretely sampled data one can either convolve the two functions
explicitly via

\begin{equation}
 \overline{f_{i}} = \frac{\sum_{u} f_{u}.s_{i-u}}
	                 {\sum_{u} s_{i-u}}
\end{equation}

or substitute a discrete Fourier transform for each of the Fourier
transforms shown in equation~\ref{convolve}. In practice, if the
width of the smoothing function encompasses a significant number of
elements of the function being smoothed then it will be faster to use the
latter approach, particularly if a Fast Fourier Transform can be used.

\subsection{Smoothing a discrete array using a weighted convolution}

If the data are weighted then one must take account of the weights
during convolution, and the convolution must be modified to

\begin{eqnarray}
 \overline{f_{i}} &=& \frac{\sum_{u} f_{u}.w_{u}.s_{i-u}} {\sum_{u} w_{u}.s_{i-u}}\\
     \overline{f} &=& \frac {f.w \otimes s} {w \otimes s}\\
     \overline{f} &=& \frac {\ift[\ft(f.w).\ft(s)]} {\ift[\ft(w).\ft(s)]}
\end{eqnarray}

The original weights of the unsmoothed data are also modified by smoothing.
If the weight of each $f_i$ is defined in terms of its variance
$\sigma_i^2$, as

\begin{equation}
  w_{i}  = \frac{1}{\sigma_{i}^2},
\end{equation}

Then the weight $\overline{w_{i}}$ that should be assigned to
$\overline{f_{i}}$, is given by:

\begin{eqnarray}
 \frac{1}{\overline{w_{i}}} &=& \sum_{j}\left(\frac{\partial\overline{f_{i}}}{\partial f_{j}}\right)^{2} . \frac{1}{w_{j}} \\
%
 \frac{\partial \overline{f_{i}}}{\partial f_{j}} &=& \frac {w_{j}.s_{i-j}}{\sum_{u} w_{u} . s_{i-u}} \\
%
 \frac{1}{\overline{w_{i}}} &=& \sum_{j}\left(\frac {w_{j}.s_{i-j}}{\sum_{u} w_{u} . s_{i-u}}\right)^{2} . \frac{1}{w_{j}} \\
%
 \label{variance}
 \frac{1}{\overline{w_{i}}} &=& \frac {\sum_{j} w_{j}.s_{i-j}^2}{(\sum_{u} w_{u} . s_{i-u})^{2}}
\end{eqnarray}

In terms of convolutions this can be re-expressed as

\begin{equation}
 \frac{1}{\overline{w_{i}}} = \frac {w \otimes s^{2} } {(w \otimes s)^2}.
\end{equation}

The smoothed data array and its weights can then be obtained via
Fourier transforms as:

\begin{eqnarray}
 \overline{f} =& \frac {f.w \otimes s} {w \otimes s} &= \frac{\ift\left[\ft(f.w).\ft(s)\right]} {\ift\left[\ft(w).\ft(s)\right]}\\
 \overline{w} =& \frac {(w \otimes s)^2}{w \otimes s^{2} } &= \frac{\left(\ift\left[\ft(w).\ft(s)\right]\right)^2}{\ift\left[\ft(w).\ft(s^{2})\right]}
\end{eqnarray}

Thus, to account for data weights when smoothing an array, one must
first determine $\ft(s)$ and $\ft(s^2)$ for the chosen smoothing
function.

\section{Smoothing functions}

\def\fwhm{\mbox{\tiny FWHM}}

Each smoothing function will be described in terms of its full-width
at half maximum value, denoted as $\fwhm$. For convenience the
dependent coordinate of each function will be defined such that it is
unity where the function falls to half its maximum value. In terms of
the spectrum channel--based coordinates, $i$ this requires the
following change of coordinates.

\begin{equation}
  \delta x = \frac{2 \delta i}{\fwhm}.
\end{equation}

In the Fourier plane this has the effect of modifying the delay
coordinate increment per channel $j$ from $1/n$ to $2.\fwhm/n$,
yielding

\begin{equation}
  \delta t = 2.\delta j.\fwhm / n.
\end{equation}

Note that in order that the total power in the smoothed data be
unchanged by smoothing. The area under the smoothing function and thus
the value at the origin of its Fourier transform, must be unity.

\subsection{Hanning}

\begin{eqnarray}
s_{x} &=& \frac{\sin(\pi x)} {2\pi x(1-x^2)}\\
\ft(s_{x}) &=& \left\{
              \begin{array}{rl} \frac{1}{2}[1 + \cos(2\pi t)] & |t| \le \frac{1}{2}\\
                                                            0 & |t|   > \frac{1}{2}\\
              \end{array}
        \right.\\
\ft(s_{x}^2) &=&  \left\{
              \begin{array}{rl} \frac{1}{4}[(1 - |t|)(1+\frac{1}{2}\cos(2\pi t)) + \frac{3}{4\pi}\sin(2\pi|t|)] & |t| \le 1\\
                                                            0 & |t|   > 1\\
              \end{array}
        \right.
\end{eqnarray}

\subsection{Gaussian}

\begin{eqnarray}
s_{x} &=& \sqrt{\ln(2)/\pi} \; e^{-\ln(2) x^{2}}\\
\ft(s_{x}) &=& e^{-\frac{\pi^2 t^2}{\ln(2)}}\\
\ft(s_{x}^2) &=& \sqrt{\ln(2)/2\pi} \; e^{-\frac{\pi^2 t^2}{2 \ln(2)}}
\end{eqnarray}

\subsection{Boxcar}

\begin{eqnarray}
s_{x} &=& \left\{
              \begin{array}{ll} \frac{1}{2} & |x| \le 1\\
                                          0 & |x|   > 1\\
              \end{array}
        \right.\\
\ft(s_{x}) &=& \frac{\sin(2\pi t)}{2\pi t}\\
\ft(s_{x}^2) &=& \frac{1}{2} \frac{\sin(2\pi t)}{2\pi t}
\end{eqnarray}

\subsection{Sinc}

\begin{eqnarray}
s_{x} &=& \frac{\beta}{\pi} \frac{\sin(\beta x)}{\beta x}\\
\beta &=& 1.8954942670340\\
\ft(s_{x}) &=& \left\{
              \begin{array}{ll} 1 & |t| \le \frac{\beta}{2\pi}\\
                                0 & |t|   > \frac{\beta}{2\pi}\\
              \end{array}
        \right.\\
\ft(s_{x}^2) &=& \left\{
                     \begin{array}{ll} \frac{\beta}{\pi} - t & |t| \le \frac{\beta}{\pi}\\
                                                           0 & |t|   > \frac{\beta}{\pi}\\
                     \end{array}
                  \right.
\end{eqnarray}

\end{document}
