\documentclass[11pt]{article}

%
% Set the page size.
%
\topmargin -0.5in \headsep 0.251in
\hoffset -0.8in
\marginparwidth 0in \marginparsep 0in
\textheight 9.5in \textwidth 6.5in
\parindent 0pt
\parskip 8pt plus 1pt minus 1pt
\pagestyle{headings}

\begin{document}

\title{Proposal to add UV-plane model fitting to Difmap.}

\author{Martin Shepherd (mcs@phobos.caltech.edu)\\
	Copyright \copyright 1994 California Institute of Technology}

\date{}
\maketitle

\clearpage

\section{Introduction}

It is common practice in VLBI to parameterize maps by fitting simple
models directly to the source UV data. These models may be used as
plausible starting models for self-calibration, or to augment the
CLEAN/self-calibration mapping loop by removing the kinds of extended
structure that CLEAN has difficulty describing.

An existing program that has been used extensively in VLBI to fit
models to UV data, is the modelfit program in the ``Caltech VLBI
Package'' \cite{tjp91}. This program is capable of fitting models
composed of aggregates of up to 15 simple components, each component
being chosen from a list that includes delta-functions, elliptical
gaussians and elliptical tapered-disks. The free parameters of the
model are adjusted to minimize the value of $\chi^{2}$ formed between
the UV-plane representation of the model and the measured amplitudes
and closure-phases. The use of closure phases rather than raw phase
makes modelfit immune to station-based phase mis-calibrations.
Modelfit does have some faults. Its two least-squares minimization
algorithms are not optimal and for the best fits and fast convergence
it relies on the coincidental match between the data and the arbitrary
iteration step sizes used. The use of closure-phases also requires
large amounts of memory and CPU time, amounts that grow as the cube of
the number of stations in the data, times the number of integrations.

While Difmap still used the same data format as the rest of the
Caltech VLBI package, no real need for a new model fitting algorithm
was deemed necessary in Difmap. But now that Difmap uses FITS as its
data format, observations that make use of the multiple dimensions
that FITS allows but the Caltech merge format does not, can not be
model fitted by the modelfit program. It is with this in mind that a
new model fitting algorithm to be incorporated in difmap is to be
written.

\section{Requirements}

Given that a new model fitting implementation is required, it would be
short sighted to simply re-produce the algorithms used in the modelfit
program without examining alternative techniques. One such technique
which purportedly incorporates the strengths of both of the algorithms
used in the modelfit program, and which has the extra advantage that
step sizes are automatically determined, is the Levenberg-Marquardt
method. This is described in \cite{nr} and will be adopted for the new
implementation.

The Levenberg-Marquardt technique requires equations for the partial
derivatives of the UV-plane model representation with respect to all
free parameters, rather than determining them numerically as the
modelfit gradient search algorithm does. The next section derives
these equations.

Unlike the modelfit program, the new algorithm will be designed to
fit models directly to the measured real and imaginary parts of the
complex visibilities. The advantage gained is mainly one of speed,
memory requirements and simplicity. In addition, the errors are more
likely to be Gaussian distributed than say those of the measured
amplitudes. The main disadvantage is that the program will not be
immune to station based phase mis-calibration. However fringe-fitting,
self-calibration and phase-referencing techniques have improved to the
extent that this should not be a significant problem.

It has been decided that the new algorithm will fit component
positions in a Cartesian rather than a polar coordinate system,
although initial guesses will still be presented in the established
polar form. This removes the problem of what to do if the model
fitting algorithm produces negative radii. The similar problem of what
to do if the sizes or aspect ratios of components go out of bounds
remains un-answered.

A useful feature that does not exist in the modelfit program, is the
estimation of the uncertainties in the fitted parameters. This is not
as straight forward as might be expected, since the free parameters
can turn out to be correlated. The full generality of this problem
will be met in the new algorithm by optionally printing out the
covariance matrix, thus leaving the onus on the user to interpret the
results.

\section{Partial derivatives of UV plane models.}

Adopting the Levenberg-Marquardt algorithm requires a knowledge of the
partial derivatives of the functions being fitted, calculated
separately with respect to each free parameter. This section will
serve to derive such equations.

A single model component of flux $S$, placed at the center of the
image plane can be represented in the UV plane by a function of form:

\begin{equation}
F(u,v) = S f(u,v)
\end{equation}

Where $(u,v)$ is a position in the UV plane. From the {\em shift
theorem} of Fourier transforms we know that the same component whose
centroid has been shifted to a position $(x,y)$ in the image plane
appears in the UV plane as:

\begin{equation}
F(u,v) = S f(u,v) e^{2 \pi i (u x + v y)}
\end{equation}

By the {\em addition theorem} of Fourier transforms we also know that
an aggregate model $M(u,v)$ can be formed from a sum of $n$ individual
components as:

\begin{equation}
M(u,v) = \sum^{n}_{i=1} S_{i} f_{i}(u,v) e^{2 \pi i (u x_{i} + v y_{i})}
\end{equation}

The partial derivative of $M(u,v)$ with respect to a single
free-parameter $\beta$ of the $k$'th component is then given by:

\begin{eqnarray}
\frac{\partial}{\partial \beta_{k}} M(u,v) & = & \frac{\partial}{\partial \beta_{k}} \sum^{n}_{i=1} S_{i} f_{i}(u,v) e^{2 \pi i (u x_{i} + v y_{i})}\\
%
\label{cmp_deriv}
\frac{\partial}{\partial \beta_{k}} M(u,v) & = & \frac{\partial}{\partial \beta_{k}} [S_{k} f_{k}(u,v) e^{2 \pi i (u x_{k} + v y_{k})}]
\end{eqnarray}

Given that for all model component types, $f(u,v)$ is independent of
flux $S$ and shifted position $(x,y)$ we can determine the following
partial derivatives for these parameters:

\begin{eqnarray}
\frac{\partial}{\partial x_{k}} M(u,v) & = & 2 \pi i u F_{k}(u,v) \\
%
\frac{\partial}{\partial y_{k}} M(u,v) & = & 2 \pi i v F_{k}(u,v) \\
%
\frac{\partial}{\partial S_{k}} M(u,v) & = & \frac{1}{S_{k}} F_{k}(u,v)
\end{eqnarray}

All the supported model types can be represented by an azimuthally
symmetric function, which is stretched into an elliptical shape along
a specific direction. In the following equations the elliptical aspect
will be parameterized by the major-axis half-extent, $a$, the
direction of the major axis with respect to North, by the angle $\phi$
measured North through East, and by the minor-axis half-extent, $b$,
or the alternative axial ratio $\gamma = b/a$.

For the purpose of model fitting we will choose to allow $a$, $\gamma$
and $\phi$ to be free parameters. The alternative choice of $a$,
$b$ and $\phi$ is less useful, because there is then no way for the
user to constrain components to have circular aspects, as required in
some specialized problems. To fit these parameters requires equations
of the partial derivatives of components with respect to them, in
addition to those above for $x$, $y$, and $S$.

To give the components the desired rotated elliptical aspect, the
value of the function at $(x,y)$ in the image plane is assigned the
value of the circularly symmetric function at radius:

\begin{equation}
R_{xy} = \sqrt{(x\cos{\phi}-y\sin{\phi})^{2}\frac{1}{\gamma^{2}} + (x\sin{\phi}+y\cos{\phi})^{2}}
\end{equation}

The Fourier transform of a circularly symmetric real function in the
image plane is another circularly symmetric function in the UV plane.
As in the image plane, the value of the rotated and stretched
function at $(u,v)$ in the UV plane is that of the circularly
symmetric function measured at UV radius:

\begin{equation}
R_{uv} = \sqrt{(u\cos{\phi}-v\sin{\phi})^{2}\gamma^{2}+(u\sin{\phi}+v\cos{\phi})^{2}}
\end{equation}

Since each of the functions of interest multiplies this radius by
$\pi a$, the following parameter will also be defined:

\begin{equation}
\Gamma = \pi a \sqrt{(u\cos{\phi}-v\sin{\phi})^{2}\gamma^{2}+(u\sin{\phi}+v\cos{\phi})^{2}}
\end{equation}

Partial derivatives of individual components that depend on $\Gamma$,
with respect to one of $a$, $\gamma$, or $\phi$ (denoted as $\beta$) are
then given by: 

\begin{equation}
\frac{\partial}{\partial \beta_{k}} M(u,v) = \frac{\partial F_{k}(u,v)}{\partial\Gamma} \times \frac{\partial\Gamma}{\partial\beta_{k}}
\end{equation}

If the latter part is evaluated for each of the desired free
parameters, this expands to:

\begin{eqnarray}
%
\label{par_phi}
\frac{\partial}{\partial\phi_{k}} M(u,v) & = & \frac{\partial
F_{k}(u,v)}{\partial\Gamma} \times \frac{2\pi^{2}a_{k}^{2}}{\Gamma} (1-\gamma_{k}^{2})
(u\cos{\phi_{k}}-v\sin{\phi_{k}})(u\sin{\phi_{k}}+v\cos{\phi_{k}}) \\
%
\frac{\partial}{\partial a_{k}} M(u,v) & = & \frac{\partial
F_{k}(u,v)}{\partial\Gamma} \times \frac{\Gamma}{a_{k}} \\
%
\frac{\partial}{\partial \gamma_{k}} M(u,v) & = & \frac{\partial
F_{k}(u,v)}{\partial\Gamma} \times \frac{\pi^{2}a_{k}^{2}\gamma_{k}}{\Gamma} 
(u\cos{\phi_{k}}-v\sin{\phi_{k}})^{2}
\end{eqnarray}

Further, since we know that neither the flux nor the position of the
components depends on $\Gamma$, we can simplify the unknown
$\frac{\partial F(u,v)}{\partial\Gamma}$ term to:

\begin{equation}
\frac{\partial F(u,v)}{\partial\Gamma} = S e^{2 \pi i (u x + v y)}
\frac{\partial f(u,v)}{\partial\Gamma}
\end{equation}

and thus only $\frac{\partial f(u,v)}{\partial\Gamma}$ need be
calculated individually for the different component types. This will
be done in the following sections.

\subsection{Gaussian components}

The equation for the value of a gaussian component at position $(u,v)$
in the UV plane is given in \cite{tjp91} as:

\begin{equation}
f(u,v) = e^{-\frac{\Gamma^{2}}{4\ln{2}}}
\end{equation}

This has a partial derivative {\em wrt} $\Gamma$ of:

\begin{equation}
\frac{\partial}{\partial\Gamma} f(u,v) = -\frac{2\Gamma}{4\ln{2}} f(u,v)
\end{equation}

\subsection{Uniformly Bright Disk components}

The equation for the value of a uniformly bright disk component at
position $(u,v)$ in the UV plane is given in \cite{tjp91} as:

\begin{equation}
f(u,v) = 2 \frac{J_{1}(\Gamma)}{\Gamma}
\end{equation}

Using a standard derivative from \cite{handbook}, the partial
derivative of $f(u,v)$ {\em wrt} $\Gamma$ is:

\begin{equation}
\frac{\partial}{\partial\Gamma} f(u,v) = -2 \frac{J_{2}(\Gamma)}{\Gamma}
\end{equation}

\subsection{Optically Thin Sphere}

The equation for the value of a optically thin sphere component at
position $(u,v)$ in the UV plane is given in \cite{tjp91} as:

\begin{equation}
f(u,v) = \frac{3}{\Gamma^{3}}(\sin{\Gamma} - \Gamma \cos{\Gamma})
\end{equation}

The partial derivative of this function {\em wrt} $\Gamma$ is:

\begin{equation}
\frac{\partial}{\partial\Gamma} f(u,v) =
\frac{9\cos{\Gamma}}{\Gamma^{3}} -
\frac{9\sin{\Gamma}}{\Gamma^{4}} +
\frac{3\sin{\Gamma}}{\Gamma^{2}}
\end{equation}

\subsection{Ring components}

The equation for the value of a ring component at position $(u,v)$ in
the UV plane is given in \cite{tjp91} as:

\begin{equation}
f(u,v) = J_{0}(\Gamma)
\end{equation}

Using a standard derivative from \cite{handbook}, the partial
derivative of $f(u,v)$ {\em wrt} $\Gamma$ is:

\begin{equation}
\frac{\partial}{\partial\Gamma} f(u,v) = -J_{1}(\Gamma)
\end{equation}

\section{Alternative parameterizations.}

The above partial derivatives were coded up for use in a simple
implementation of the Levensburg-Marquardt non-linear least-squares
algorithm. It turned out that the parameterization of the elliptical
part of the models had significant problems. In particular in
equation~\ref{par_phi} the partial derivative of the model {\em wrt}
$\phi$ is zero when $\gamma_{k}=1$. This means that the parameter can
be changed by an infinite amount without changing the model. This
results in zero column and row vectors in the Levensburg-Marquardt
Hessian matrix, which in turn results in a singular matrix when
inverted.

A better parameterization that is free of this problem was initially
suggested by Steve Myers and formalized by Tim Pearson. The essential
parts of Tim Pearson's suggestion are included below.

\subsection{Reformulation of Model Fitting [Written by Tim Pearson]}

I start from Martin's definition (11) of parameter $\Gamma$ for an
elliptical component with major axis $a$, minor axis $b= \gamma a$,
and position angle $\phi$: 

\begin{equation}
\Gamma = \pi \sqrt{a^2(u \sin\phi + v\cos\phi)^2 + b^2(u\cos\phi -
v\sin\phi)^2}.
\end{equation}

Expanding this expression and collecting terms,

\begin{equation}
\Gamma^2/\pi = \cos^2\phi(a^2v^2 + b^2u^2) +
               \sin^2\phi(a^2u^2 + b^2v^2) + 2uv\cos\phi\sin\phi(a^2-b^2).
\end{equation}

Applying the double-angle formulae:

\begin{eqnarray}
\sin2z & = & 2\sin z \cos z, \\ 
\cos2z & = & 2\cos^2 z -1 = 1 - 2\sin^2 z, 
\end{eqnarray}

we obtain

\begin{eqnarray} 
\Gamma^2/\pi &=& (1 + \cos 2\phi)(a^2v^2 + b^2u^2)/2 + (1 - \cos 2\phi)(a^2u^2 + b^2v^2)/2 + uv \sin 2\phi (a^2 - b^2) \\
%
             &=& (a^2+b^2)(u^2+v^2)/2 + \cos 2\phi(a^2-b^2)(v^2 -u^2)/2 + uv\sin 2\phi (a^2 - b^2) \\
%
             &=& (v^2 -u^2) X + 2uv Y + (u^2+v^2) Z,
\end{eqnarray}

where I have defined 

\begin{eqnarray}
 X &=& (a^2 -b^2)\cos 2\phi /2,\\
 Y &=& (a^2 -b^2)\sin 2\phi /2,\\
 Z &=& (a^2+b^2)/2.
\end{eqnarray}

A point in the $(X,Y,Z)$ system represents an ellipse. Circles ($a=b$)
are on the line $X=Y=0$.

\subsection{Partial derivatives.}

We need the derivatives of

\begin{equation}
\Gamma = \pi \sqrt{(v^2 -u^2) X + 2uv Y + (u^2+v^2) Z}
\end{equation}

with respect to the model parameters $X,Y,Z$:

\begin{eqnarray}
\frac{\partial\Gamma}{\partial X} &=& \frac{\pi^2}{2\Gamma} (v^2 - u^2),\\
\frac{\partial\Gamma}{\partial Y} &=& \frac{\pi^2}{2\Gamma} (2uv),\\
\frac{\partial\Gamma}{\partial Z} &=& \frac{\pi^2}{2\Gamma} (v^2 + u^2).
\end{eqnarray}

\subsection{Disadvantages}

Use of $(X,Y,Z)$ as variable parameters instead of $(a,\gamma,\phi)$
has the disadvantage that axial ratio and position angle cannot be
constrained. Fixing $Z$ is somewhat equivalent to fixing major
axis. There are some non-physical domains in $X,Y,Z$ space, e.g.,
$Z<0$.

\subsection{Advantages}

$X,Y,Z$ all have the same dimensions so equal increments in each parameter
are likely to affect $\chi^2$ by similar amounts. Components can be
constrained to be circular by fixing $X=Y=0$. Note that convolving two
elliptical gaussians is equivalent to adding their $X,Y,Z$ parameters---not
that this has any relevance to model fitting.


\begin{thebibliography}{xx}
\bibitem{handbook} Abramowitz, M. and Stegun, I.A (eds) 1972 {\em
Handbook of Mathematical functions} \S~9.1.30, 9th printing, Dover publications

\bibitem{tjp91} Pearson, T.J. 1991 {\em Introduction to the Caltech VLBI
Programs} \S~7.2

\bibitem{nr} Press, W. H., Flannery, B. P., Teukolsky, S. A. and
Vetterling, W. T. 1989, {\em Numerical Recipes}, Cambridge University
Press.
\end{thebibliography}
\end{document}
